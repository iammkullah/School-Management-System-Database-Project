\chapter{Literature Review}
\label{chp2}
In the chapter 1 we have given the introduction of our project, motivation behind doing that, project scope, Features of Web-Based School Management
System and a report break down. Our introduction chapter is giving a complete overview of this project. This chapter is about studies and literatures that are related to the School Management that extended the knowledge related to our project idea, guide us to improve and develop our proposed system more effectively.
\section{Literature Survey}
This is a new attempt to speed up the process of managing data in an educational institute. The existing systems are time-consuming and there are many difficulties faced by the administrator to get information about each and everyone within the organization. Presently in many institutions, most of the tasks are carried on manually such as Employee Registration, Student List, Employee List, Student Attendance, Employee Attendance, Student Routine, Result Management etc,. There are many difficulties faced by instructors, parents, administration, and students for carrying out data related to different activities. 
\section{Introduction to Database Management System}
DBMS stands for Database Management System. We can break it like this DBMS= Database +Management System. Database is a collection of data and Management System
is a set of programs to store and retrieve those data. Based on this we can define DBMS like this: DBMS is a collection of inter-related data and set of programs to store and access those data in an easy and effective manner.
Database system are basically developed for large amount of data. When dealing with huge amount of data, there are two things that require optimization: Storage of data and retrieval of data.\\\\
According to the principles of database systems, the data is stored in such a way that it acquires a lot less space as the redundant data(duplicate data)
has been removed before storage. Along with storing the data in an optimized and systematic manner, It is also important that we retrieve the data quickly when needed. Database system ensures that data is retrieved as quickly as possible.
\section{Applications of Database Management System}
The development of computer graphics has been driven both by the needs of
the user community and by the advances in hardware and software. The applications
of database are many and varied; it can be divided into four major areas:
\begin{enumerate}
\item Hierarchical and network system
\item Flexibility with relational database
\item Object oriented application
\item Interchanging the data on the web for e-commerce
\end{enumerate}
